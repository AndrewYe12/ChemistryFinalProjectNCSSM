\documentclass[../main.tex]{subfiles}
\usepackage{graphicx} % Required for inserting images
\usepackage{chemfig}
\usepackage{chemformula}
\usepackage[version=4]{mhchem}
\usepackage{modiagram}
\usepackage{gensymb}
\usepackage{multirow}
\usepackage{amsmath}
\usepackage{lewis}
\graphicspath{{\subfix{../photos/}}}


\begin{document}
\newcommand{\countThis}{\thenooCounter \stepcounter{nooCounter}}
\newcommand{\AnswerSet}{\subsubsection*{Problem \countThis}}
\newcounter{nooCounter}
\setcounter{nooCounter}{1}
\newpage
\section{Answers}
\subsection{Unit 1}
\subsubsection*{Problem \countThis}
The molar mass of \(\ce{C_9H_8O_4}\) is \(1.008*8 + 12.01*9 + 16.00*4 = 180.2 \frac{g}{mol}\)
\begin{equation}
\begin{aligned}
    7.89kg \times \frac{1g}{10^{-3}kg} \times \frac{1mol}{180.2g} = 43.8mol
\end{aligned}
\end{equation}
\subsubsection*{Problem \countThis}
(b), the tallest peak of the graph is the one at \(64u\). 
\subsubsection*{Problem \countThis}
In one mole of \(\ce{C_{13}H_{18}O_2}\) is \(206.31g\).
\begin{equation}
\begin{aligned}
    1mol\hspace{0.1em}\ce{C_{13}H_{18}O_2} \times \frac{13mol\hspace{0.1em}\ce{C}}{1mol\ce{C_{13}H_{18}O_2}} \times \frac{12.01g}{1mol\hspace{0.1em}\ce{C}} = 156.31g
\end{aligned}
\end{equation}
Thus, the percent composition by weight is \(\frac{156.31}{206.31} = 75.764\%\)
\subsubsection*{Problem \countThis}
Take \(100g\) of the substance such that there are \(32.38g\) sodium, \(22.65g\) sulfur, and \(44.99g\) oxygen. 
\begin{equation}
\begin{aligned}
    32.38g\hspace{0.1em}\ce{Na} \times \frac{1mol\hspace{0.1em}\ce{Na}}{22.99g} &= 1.408mol\hspace{0.1em}\ce{Na} \\
    22.65\hspace{0.2em}g \ce{S} * \frac{1\hspace{0.2em}mol\ce{S}}{32.07g} &= 0.7063 \hspace{0.2em}mol\ce{S}\\
    44.99\hspace{0.2em}g\ce{O} * \frac{1\hspace{0.2em}mol\ce{O}}{16 g} &= 2.812\hspace{0.2em}mol \ce{O}
\end{aligned}
\end{equation}
Take the ratio of each compound with the smallest quantity. 
\begin{equation}
\begin{split}
    S:\hspace{0.2em}\frac{0.7063}{0.7063} = 1 \\
    Na:\hspace{0.2em}\frac{1.408}{0.7063} = 2 \\
    O:\hspace{0.2em}\frac{2.812}{0.7063} = 4
\end{split}
\end{equation}
Therefore, the empirical formula is \(Na_{2}SO_{4}\)
\subsubsection*{Problem \countThis}
\(1s^22s^22p^63s^23p^64s^23d^{10}4p^65s^24d^{10}5p^66s^24f^{14}5d^{10}\)
\subsubsection*{Problem \countThis}
\(\ce{Be}\). The peak location of the peak on the x-axis means that there is less binding energy for the electrons in element \(X\). \(\ce{Be}\) has fewer protons
and both electrons are in the same shell, so it peak must belong to \(\ce{Be}\).
\subsubsection*{Problem \countThis}
\begin{itemize}
\item The electronegativity increases from left to right across a period. This is because if a valence shell of electrons is less than half full than
it requires less energy to lose an electron than gain one. If if the valence shell of electrons is more than half full, it is easier to pull an electron
into the valence shell. The electronegativity decreases from the top to the bottom of a group. This is beause there is a greater atomic radius lower on the group.

\item The ionization energy increases from left to right in a period. This is because of greater valence shell stability also because of smaller atomic radius. The ionization energy also decreases
from top to bottom of a group. This is because of greater electron shielding and greater atomic radius. 

\item Atomic radius decreases from left to right within a period. This is because there are more protons to the right of the period. Atomic radius increases from top to bottom within a group.
 This is because of electron shielding and there are more electron shells in the atom. 
\end{itemize}  
\subsection{Unit 2}
\subsubsection*{Problem \countThis}
The ionic character increases the greater the electronegativity difference. In this case, \(\ce{Na}\) and \(\ce{O}\) had the greatest electronegativity difference. 
\subsubsection*{Problem \countThis}
(c) \(\chemfig{Cl-F} > \chemfig{H-I} > \chemfig{Se-N}\)
\subsubsection*{Problem \countThis}
\ce{Cs+} has a larger atomic radius than \ce{K+}. So the distance between the cation and anion is greater than in \ce{CsF} than in \ce{KF}
\subsubsection*{Problem \countThis}
Most metallic elements form crystalline solids at room temperature. Their bonds are metallic bonds due to electrostatic attraction between metal cations and delocalized electrons. 
\subsubsection*{Problem \countThis}
\begin{itemize}
    \item Substitutional alloys. These alloys form when one atom of a similar size to the host metal replaces an atom of the host metal. The substitute atom must be of similar size. These alloys have good thermal and electrical conductivity.
    \item Interstitial alloys. These alloys are formed when smaller atoms fill in the gaps between the larger host atoms. This makes the metal harder and less malleable. 
\end{itemize}
\subsubsection*{Problem \countThis}
\chemfig{H-C(-[:90]H)(-[:270]H)-C(=[:90]\charge{180=\:, 0 = \:}{O})-\charge{90 =\:, -90 =\:}{O}-H}
\subsubsection*{Problem \countThis}
\chemfig{\charge{90=\:,-90=\:}{O}=C=\charge{90 =\:, -90=\:}{O}}
\subsubsection*{Problem \countThis}
\chemfig{\charge{90=\:}{O}(=[:330]\charge{60=\:,240=\:}{O})(-[:210]\charge{120=\:, 315=\:, 225=\:}{O})}
\(\longleftrightarrow\)
\chemfig{\charge{90=\:}{O}(=[:210]\charge{135=\:,315=\:}{O})(-[:330]\charge{45=\:, 315=\:, 225=\:}{O})}
\AnswerSet
All formal charges of \ce{CH3COOH} and \ce{CO2} are zero. \\
\chemfig{O^{+1}(=[:330]O)(-[:210]O^{-1})}
\(\longleftrightarrow\)
\chemfig{O^{+1}(=[:210]O)(-[:330]O^{-1})}
\AnswerSet
The electron geometry is tetrahedral. The molecular geometry is trigonal pyramidal. Hybridization of \ce{N} atom is \(sp^3\) since it has tetrahedral electron geometry.
\subsection{Unit 3}
\AnswerSet
Dipole-diple and london dispersion forces. The \chemfig{C-O} bond is polar and the molecule is asymmetrical so it is polar. There are no \chemfig{H-F}, \chemfig{H-O}, or \chemfig{H-N} bonds, so there is no hydrogen bonding. 
\AnswerSet
Condensation. Both have no regular arrangement, the one on the left is separated by the one on the right is close together, so the molecules are transitioning from gas to liquid.
\AnswerSet
(a) As you increase the temperature, the average kinetic energy or speed increases as well. \\
(b) \(n_1 = n_2\) and \(V_1 = V_2\), so \(P_1V_1 = n_1RT_1\) and \(P_2V_2 = n_2RT_2\). \(\frac{P_2V_2}{RT_2} = \frac{P_1V_1}{RT_1} \implies P_2 = \frac{P_1T_2}{T_1} = \frac{0.7 * 425}{299} = 0.99atm\)\\
(c) As the temperature increases, the average kinetic energy increases, so the molecules undergo more collisions with the walls of the container.  \\
\AnswerSet
\begin{equation*}
    60.3g \times \frac{1mol\hspace{0.2em}\ce{Ba(OH)2}}{171.35g} \times \frac{2mol\hspace{0.2em}\ce{OH-}}{1mol\hspace{0.2em}\ce{Ba(OH)2}} * \frac{1}{1.75L} = 0.402M
\end{equation*}
\AnswerSet
The photoelectric effect occurs when light of a certain minimum frequency/energy hits the surface of a metal and electrons are ejected. 
\AnswerSet
\(\ce{Ar} < \ce{H2S} < \ce{HCOOH}\)
\AnswerSet
London dispersion forces. Benzene is symmetrical so there are no dipole dipole forces.
\AnswerSet
\ce{MgF2} 
\AnswerSet
Because \ce{I2} has a more polarizable electron cloud. 
\AnswerSet
No, it is symmetrical.
\subsection{Unit 4}
\AnswerSet
\ce{2C5H_{10} + 15O2 -> 10CO2 + 10 H2O}
\AnswerSet
The oxidation reaction: \\
\begin{equation*}
\begin{aligned}
& \ce{I- -> I2} \\
& \ce{2I- -> I2}\\
& \ce{2I- -> I2 + 2e-} \\
& \ce{10I- -> 5I2 + 10e-}
\end{aligned}
\end{equation*}
The reduction reaction:
\begin{equation*}
    \begin{aligned}
        & \ce{MnO4- -> Mn^{2+}} \\
        & \ce{MnO4- -> Mn^{2+} + 4H2O}\\
        & \ce{MnO4- + 8H+ -> Mn^{2+} + 4H2O} \\
        & \ce{5e- + MnO4- + 8H+ -> Mn^{2+} + 4H2O} \\
        & \ce{10e- + 2MnO4- + 16H+ -> 2Mn^{2+} + 8H2O}
    \end{aligned}
\end{equation*}
Add the reactions together: \\
\ce{10I- + 16H+ + 2MnO4- -> 5I2 + 2Mn^{2+} + 8H2O}
\AnswerSet
\ce{Fe^{3+}(aq) + 3OH-(aq) -> Fe(OH)3(s)}
\AnswerSet
Chemical processes are characterized by changes in intramolecular forces, while physical processes are characterized by changes only in intermolecular forces.
\AnswerSet
Find the limiting reactant:
\begin{equation}
    \begin{split}
        36.0\hspace{0.2em}gH_{2}O * \frac{1\hspace{0.2em}molH_{2}O}{18.02\hspace{0.2em}gH_{2}O} *\frac{1\hspace{0.2em}molFe_{3}O_{4}}{4\hspace{0.2em}molH_{2}O} = 0.49945 \hspace{0.2em}mol Fe_{3}O_{4}\\
        67.0\hspace{0.2em}gFe * \frac{1\hspace{0.2em}molFe}{55.85\hspace{0.2em}gFe} * \frac{1\hspace{0.2em}mol Fe_{3}O_{4}}{3\hspace{0.2em}molFe} = 0.39988\hspace{0.2em}mol Fe_{3}O_{4}
    \end{split}
\end{equation}
Therefore, the limiting reactant is \ce{Fe} \\
Find how much iron oxide is produced: Use the limiting reactant
\begin{equation}
    0.39988\hspace{0.2em}mol \ce{Fe} * \frac{231.55\hspace{0.2em}gFe_{3}O_{4}}{1\hspace{0.2em} mol\ce{Fe}} = 92.6 \hspace{0.2em}g Fe_{3}O_{4}
\end{equation}
Find how much excess reactant is left over:
\begin{equation}
    67.0\hspace{0.2em}gFe * \frac{1\hspace{0.2em}molFe}{55.85\hspace{0.2em}gFe} * \frac{4\hspace{0.2em}mol H_{2}O}{3\hspace{0.2em}molFe} * \frac{18.02\hspace{0.2em}gH_{2}O}{1\hspace{0.2em}molH_{2}O} = 28.8\hspace{0.2em}g H_{2}O
\end{equation}
28.8 grams of water is used out of 32.0 grams. So there is 7.2 grams left over of the excess reagent. 
\AnswerSet
\ce{4Al(s) + 3O2(g) -> 2Al2O3(s)}
\AnswerSet
\ce{4V + 5O2 -> 2V2O5}
\AnswerSet
\(\ce{C6H_{12}O6 + 6O2 -> 6CO2 + 6H2O}\)
\AnswerSet
\ce{2Mg + O2 -> 2MgO}
\subsection{Unit 5}
\AnswerSet
\(1.5 \times 10^3mol\) 
\AnswerSet 
\ce{CH4} \hspace{0.5em} rate = \(\frac{-\Delta [\ce{CH4}]}{\Delta t}\) \\
\ce{O2} \hspace{0.5em} rate = \(\frac{-1}{2} \frac{\Delta [\ce{O2}]}{\Delta t}\) \\
\ce{CO2} \hspace{0.5em} rate = \(\frac{\Delta [\ce{CO2}]}{\Delta t}\) \\
\ce{H2O} \hspace{0.5em} rate = \(\frac{1}{2} \frac{\Delta [\ce{H2O}]}{\Delta t}\)
\AnswerSet
\(10\frac{M}{s}\)
\AnswerSet
\(\frac{(0.04 - 0.1)M}{125ms} \times \frac{1ms}{10^{-3}s} = -0.48 \frac{M}{s}\)\\
rate = \(-\frac{1}{2} \times -0.48 = 0.24 \frac{M}{s}\)
\AnswerSet
When concentration of \(B\) is held constant and the concentration of \(A\) is tripled, the initial rate is multiplied by \(9\) so the order with respect to \(A\)
is \(2\). The order with respect to \(B\) is \(0\) because nothing changes when the concetration is increased. Thus the rate law is 
\begin{equation*}
    rate = k[A]^2[B]^0 = k[A]
\end{equation*}
\(1.0 \times 10^{-2} = k \times 0.1^2 \implies k = 1.0 \frac{1}{ms}\), thus \(rate = [A]^2\)
\AnswerSet
\begin{equation*}
    \begin{aligned}
    ln[A]_t &= -kt + ln[A]_0 \\
    ln[A]_t &= -(4.8 \times 10^{-4})(825) + ln(0.0165)\\
    [A]_t = e^{-4.50} = 0.0111M 
    \end{aligned}
\end{equation*}
Half-life:
\begin{equation*}
    \begin{aligned}
        ln(\frac{1}{2}[A]_0) - ln[A]_0 &= -kt \\
        ln(\frac{1}{2}) &= -kt \\
        t &= \frac{ln(2)}{k}
    \end{aligned}
\end{equation*}
\AnswerSet
\(k = 4.0 \times 10^{-4} \frac{1}{Ms} = 35\frac{1}{M\times days}\) \\
\(\frac{1}{[A]_t} = 35 * 6 + \frac{1}{0.1} \implies [A]_t = 4.5 \times 10^{-3}\)
\AnswerSet
\(\frac{1}{0.085M} = 35t + \frac{1}{0.1M} \implies t = 0.05days\)
\AnswerSet
\ce{2H2O2 ->  2H2O + O2} 
The rate determining step is the slow reaction, so \(rate = k[\ce{H2O2}][\ce{I-}]\)\\
\ce{I-} is the catalyst. \ce{IO-} is the intermediate. 
\AnswerSet
Overall: \(\ce{2NO + 2H2 -> 2H2O + N2} \)\\
Rate determining step: \(rate = k[\ce{N2O2}][\ce{H2}]\) \\
Fast equilbium: \(\text{rate forward} = \text{rate back} \implies [\ce{N2O2}] = \frac{k_f}{k_r}[\ce{NO}]^2\) \\
Thus, \(rate = \frac{kk_f}{k_r}[\ce{NO}]^2[\ce{H2}]\)
\subsection{Unit 6}
\AnswerSet
\(50g\ce{H2O}\times \frac{1mol\ce{H2O}}{18.01g\ce{H2O}} \times \frac{6.022\times 10^{23}\ce{H2O}}{1mol\ce{H2O}} \times \frac{2\ce{O-H}}{1\ce{H2O}} 
\times \frac{1.8 \times 10^{-9}cal}{1\ce{O-H}} \times \frac{4.184J}{1cal}\times\frac{1kJ}{10^3J} = 2500kJ\)
\AnswerSet
\(q_{water} = 7.3 \times 100 \times 4.184\) \\
\(q_{water} = -q_{metal}\)\\
\(q_{metal} = C_{metal} (27.3-100)*120 \implies C_{metal} = 0.350 \frac{J}{g\degree C}\)
\AnswerSet
\(5.0L\ce{H2O} \times\frac{1mL\ce{H2O}}{10^{-3}L\ce{H2O}}\times\frac{1g\ce{H2O}}{1mL}\times\frac{1mol}{18.01g}\times\frac{40.72kJ}{1mol} = 11302.32kJ\) 
\AnswerSet
\begin{equation*}
\begin{aligned}
& \ce{C2H2 + 2H2 -> C2H6} \\
& \ce{C2H6 -> C2H4 + H2} 
& \implies \ce{C2H2 + H2 -> C2H4}
\end{aligned}
\end{equation*}
Thus, \(\Delta H = -175\)
\AnswerSet
\(1.5mol\ce{O2} \times \frac{1mol_{rxn}}{3mol\ce{O2}} \times \frac{-1371kJ}{mol_{rxn}} = -685.5kJ  \) 
\AnswerSet
Increase
\AnswerSet
\(\Delta S = 205.0 + 2\times 130.58 - 2 \times 188.83 = 88.5 \frac{J}{K}\)
\AnswerSet
\(\Delta S_{surr} = \frac{-\Delta H}{T} = -\frac{-802.2*1000}{298.15} = 2691 \frac{J}{K}\) \\
\(\Delta S_{sys} = 2\times 188.7 + 213.7 - (186.1 + 2\times 205.0) = -5.0 \frac{J}{K}\) \\
\(\Delta S_{universe} = -5 + 2691 = 2686\frac{J}{K}\). 
\AnswerSet
\(\Delta G = -19800J - (1000)*(-197.3\frac{J}{K}) = 106000J\) 
\AnswerSet
More product at high temperature 
\subsection{Unit 7}
\AnswerSet
\(K_c = \frac{[\ce{CH4}][\ce{H2O}]}{[\ce{H2}]^2[\ce{CO}]}\)
\AnswerSet
\(Q_c = 6.25\). \(Q_c > K_c\) so the reaction will shift towards reactants. 
\AnswerSet
\(K_p =\frac{(3x)^3x}{(14-2x)^2} \implies 2.0 \times 10^{-6} \approx \frac{(3x)^3x}{14^2} \implies x = 0.062atm\). The partial pressure is \(0.062atm\). 
\AnswerSet
\(x^2 = 0.16 \implies x = 0.4M\). Note that \(250g = 4.89mol\) which is clearly enough to produce the \(0.4M\) predicted by the equilibrium constant. 
\AnswerSet
\(K_{sp} = [\ce{Ag+}][\ce{Cl-}] = 1.8 \times 10^{-10} \implies x^2 = 1.8\times 10^{-10} \implies x = 1.3\times 10^{-5}M\)
\AnswerSet
\(K_{sp} = [\ce{Ca^{2+}}]^3[\ce{PO4^{3-}}]^2 \implies 108x^5 = 1.2\times 10^{-29} \implies x = 6.4\times 10^{-7}M\)
\AnswerSet
\(x = 15mg \times \frac{10^{-3}g}{1mg}\times \frac{1mol}{78.08g} = 1.9\times 10^{-4}M\) \\
\(K_{sp} = [\ce{Ca^{2+}}][\ce{F-}]^2 = x(2x)^2 = 2.7\times 10^{-11}\)
\AnswerSet
\(1.0\times 10^{-14} = 1.3[\ce{OH-}] \implies [\ce{OH-}] = 7.7\times 10^{-15}M\)
\AnswerSet
\(pH = -log(0.05) = 1.30\)
\AnswerSet
\begin{tabular}{||c|c|c|c||}
\hline 
\ce{HA} & \ce{H2O} & \ce{H3O+} & \ce{A-} \\[0.5ex] 
\hline \hline
0.2 & & \(\sim 0\) & 0 \\
\(-x\) & & x & x \\
\(0.2-x\) & & \(\sim x\) & x \\[1ex]
\hline
\end{tabular}\\
\(2.6\times 10^{-5} = \frac{x^2}{0.2-x} \approx \frac{x^2}{0.2} \implies x = 0.00228 \implies pH = 2.64\)
\subsection{Unit 8}
\AnswerSet
\(pH = pK_a + log(\frac{[\ce{CH3COO}]}{[\ce{CH3COOH}]}) \implies  pH = -log(1.8\times 10^{-5}) + log(\frac{0.1}{0.5}) \implies pH = 4.04\)
\AnswerSet
\(pH = -log(1.8\times 10^{-5}) + log(\frac{0.5}{0.1}) = 5.44\)
\AnswerSet
(c)
\AnswerSet
Yes
\AnswerSet
Yes, the strong base reacts with half of the weak acid to produce weak base. 
\AnswerSet
\([\ce{H3O+}] = 10^{-3.7} = 2.0 \times 10^{-4}\)\\
Percent ionization = \(\frac{2.0\times 10^{-4}}{0.1}\times 100 = 0.2\%\)
\AnswerSet
\ce{HF} is weak even though \ce{F} is highly electronegative because the bond between \ce{HF} is stronger than the bond between \ce{HCl}.
\AnswerSet
Because the highly electronegative \ce{O} in the molecule of \ce{HClO4} pulls away electrons more effectively than the \ce{HClO2} because \ce{HClO2} only has 2 electrons. Thus the \chemfig{H-O} bond is more polar in \ce{HClO4}
\AnswerSet
\(\ce{SO4^{2-}}\)
\AnswerSet
\(1 \times 10^{-7}\)
\subsection{Unit 9}
\AnswerSet
Chlorine is oxidized.
\AnswerSet
Reduction: \(\ce{2Al^{3+} + 6e- -> 2Al}\)\\
Oxidation: \(\ce{3Mg -> 6e- + 3Mg^{2+}}\)
\AnswerSet
\(Q = \frac{[\ce{Mg^{2+}}]^3}{[\ce{Al^{3+}}]^2}\)\\
So \(E_{cell} = E_{cell}\degree -\frac{RT}{nF} ln(Q) = 0.71 - \frac{8.314\times 298}{6\times 96485}\times ln(1\times 10^{4}) = 0.67V\).
\AnswerSet
\(\ce{Al^{3+} + 3e- -> Al}\)\\
\(q = I\times t = 10 C\) \\
\(n = \frac{q}{F} = \frac{10C}{96485\frac{C}{mol}} = 1.04\times 10^{-4} mol\ce{e-}\) \\
\(1.04\times 10^{-4}mol\ce{e-}\times \frac{1mol\ce{Al^{3+}}}{3mol\ce{e-}}\times \frac{26.98g}{1mol} = 9.32\times 10^{-4}g\)

\end{document}
